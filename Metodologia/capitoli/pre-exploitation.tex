\chapter{Pre-Exploitation}
\markboth{Pre-Exploitation}{}

\section{Target Scoping}
In questa fase bisogna stipulare un accordo tra le parti (responsabile dell'asset e pentester) in modo da definire vincoli, limiti, responsabilità legali in caso di eventuali problemi, accordo di non divulgazione, ecc. Tuttavia, possiamo fare le seguenti osservazioni:

\begin{itemize}
    \item L'asset da analizzare è pubblicamente disponibile e realizzato appositamente per essere analizzato, ossia vulnerabile by-design;
    \item Tutta l'analisi avviene in un ambiente virtualizzato all'interno della macchina in possesso al Penetration Tester;
    \item Lo scopo dell'analisi è puramente didattico, in quanto realizzato in un contesto universitario e, più precisamente, come progetto del corso "Penetration Testing and Ethical Hacking";
    \item Tutti gli strumenti utilizzati e le fonti consultate sono pubblicamente disponibili e accessibili o, in generale, sono accessibili tramite piani gratuiti e quindi senza costi da sostenere.
\end{itemize}

In conclusione, come si può notare dalle precedenti osservazioni, questa fase può essere tranquillamente saltata visto che non ci sono parti con cui prendere accordi e non possono esserci problematiche di tipo legale dal momento che l'ambiente è totalmente simulato.

\section{Information Gathering}
Durante questa fase, l'obiettivo è quello di trovare più informazioni possibili riguardo l'asset scelto e, essendo che l'asset è una macchina virtuale che viene eseguita in un \emph{ambiente virtualizzato} e in una \emph{rete con NAT virtuale} (come illustrato nell'introduzione), si eviteranno fonti e tool che raccolgono informazioni riguardo persone afferenti all'organizzazione dell'asset, indirizzi e-mail, analisi di record DNS, informazioni di routing e così via. A questo punto, l'unica tecnica che ha senso utilizzare (e che è stata effettivamente utilizzata) è \textbf{OSINT} (\emph{Open Source INTelligence}), con cui si cercherà di individuare nomi utente, password, indirizzo IP, ecc. Tutto questo, ovviamente, evitando di consultare fonti dove sono presenti Walktrough e guide per evitare di vanificare il contributo didattico del processo.

Come primo passo, è stata consultata la pagina di Vulnhub sulla quale sono riportate varie informazioni riguardo la macchina virtuale scelta "\textbf{De-ICE S1.140}" e,
all'interno della pagina, sono state trovate le seguenti informazioni:
\begin{itemize}
    \item Informazioni riguardo il \textbf{rilascio}, ovvero autore, data, sorgente e valore hash della macchina. Queste informazioni, tuttavia, non sembrano essere utili per il processo;
    \item Una \textbf{descrizione} molto ad alto livello della macchina. Anche qui non viene rilasciata alcuna informazione utile come servizi esposti dalla macchina o credenziali di accesso alla macchina (anche non privilegiate). Infatti, attualmente, se avviamo la macchina non possiamo fare nulla tramite \emph{interazione diretta} in quanto \textbf{non abbiamo nessuna credenziale di accesso};
    \item Informazioni riguardo la configurazione dell'\textbf{indirizzo di rete}. Questa informazione è molto utile perché ci rivela che la macchina \textbf{non è configurata per lavorare con un indirizzo IP specifico} ma lo ottiene in maniera automatica grazie al servizio \textbf{DHCP}. Questo ci fa subito capire che all'interno della rete con NAT non avremo problemi di indirizzamento ma, sfortunatamente, questo significa che non conosciamo apriori l'indirizzo della macchina (sappiamo solo che sarà all'interno della rete \emph{10.0.2.0/24}) ma dovremo ricavarcelo in maniera indiretta visto che \emph{VirtualBox} non fornisce un metodo diretto di ottenimento degli indirizzi IP e \textbf{non abbiamo accesso alla macchina};
    \item Informazioni riguardo il \textbf{sistema operativo}. Altra informazione molto utile in quanto adesso sappiamo che l'asset è un sistema Linux e questo ci permetterà di risparmiare tempo in fasi avanzate perché possiamo restringere il campo delle scansioni solo a sistemi Linux, escludendo tutti gli altri. Tuttavia, non sappiamo ancora la versione precisa del kernel e quindi dobbiamo ricavarla successivamente;
\end{itemize}

Andando più a fondo nella pagina si può ricavare l'indirizzo del \textbf{sito web del creatore} dell'asset e il \textbf{link di download della macchina} ma, sfortunatamente, entrambi i link \textbf{non sono più attivi}. Consultando il motore di ricerca Google, semplicemente ricercando il nome dell'organizzazione trovato sulla pagina, è possibile risalire al rispettivo account Twitter e si nota che quest'ultimo non è attivo all'incirca dal 2020. Per questa ragione, è sembrato opportuno accedere al servizio \emph{WaybackMachine} offerto da \textbf{Archive.org} per visitare versioni precedenti del sito dell'organizzazione nella speranza di trovare altre informazioni utili. Fortunatamente, grazie a questo servizio è stato possibile accedere ad uno \emph{snapshot} risalente al 2021 dal quale è stato anche possibile effettuare il download della macchina. Ad ogni modo, anche accedendo al sito e, in particolare, alla pagina di download, non sono state trovate informazioni rilevanti come credenziali, porte aperte, schemi di naming, ecc.

\section{Target Discovery}

\section{Target Enumeration}